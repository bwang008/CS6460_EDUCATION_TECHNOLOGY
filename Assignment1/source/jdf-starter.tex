\documentclass[
	%a4paper, % Use A4 paper size
	letterpaper, % Use US letter paper size
]{jdf}

\addbibresource{ref.bib}

\author{Benjamin Wang}
\email{bwang421@gatech.edu}
\title{CS 6460 Assignment 1}

\begin{document}
%\lsstyle

\maketitle

\begin{abstract}
	This is a paper to document the research and journey which I am taking in order to create a brief online course.
\end{abstract}

\section{Journal}
\subsection{Introduction}
Hello, my name is Benjamin Wang and this is my first journal for CS6460 Education Technology! I am thrilled to be taking this class and really looking forward to producing some great content to teach others and hopefully make an impact on the learning community.

A little background on myself, I am a 31 year old electrical engineer living in Los Angeles working in the power industry who has always had a love for learning algorithms, building data processes, and models. 

Throughout my journey, I've worked through a lot of different learning materials from books, videos, online classes, and website tutorials and from all of these materials I've found myself wishing I could one day become a contributor to the field of online learning through my preferred medium: Video. 

I've had a lot of experience teaching, tutoring, and training people in personal settings as it's a setting I'm very comfortable with, but I know that other skills and aspects are involved in constructing a great course which is why I saw this class as a great opportunity to knock out two birds with one stone!

There are several challenges which I've never faced before with this class, first off I am not one to read academic papers often, let alone 15 in a week so I was shocked by that number! But the research guide on reading papers was very eye-opening in terms of how I should approach the task. 

Another task is using latex/overleaf. I was first introduced to latex last year and have been trying to use every opportunity available to learn the syntax and capabilities of the language, so I'm really looking forward to expanding my knowledge and skill in this area. 

\subsection{Review of Past Projects}
The first thing I did was browse through the different content projects on canvas and explored the content other students created which the professors found worth highlighting. 

One of the papers written by Mark Hess \cite{hess1} talks about how he decided to design a class for motorcycle owners on how to ride despite being a novice in the area because he felt that many of the existing resources had been created by experts in the field, and as a total beginner he felt overwhelmed with the amount of information, concepts, and terminology that the teachers assumed he was already familiar with.

I really liked this concept of course design, and it falls along the lines of the area which I want to contribute in since I also am not an expert in data science.

My initial thoughts for my data science course was to target an audience that is familiar with working with platforms like Microsoft Excel, but also interested in expanding their analysis skills to incorporate things like VBA macros and python scripts.

I was thinking of creating my own original content in terms of problem sets, and additional explanation videos, but it seems like this was unrealistic due to time constraints, so now I am considering the route of sourcing other videos in terms of explanations, or considerably reducing the size of what content I was hoping to create. 

One reason I wanted everything to be self-contained as possible was for metrics. By creating all of the content, I can track the progress of students through clicks and see where they are going.

Another project also worked on producing a course \citep{guida1} and went into great detail on the tools used for video editing and production. 

\subsection{Youtube}
The reason I opted for Youtube as the hosting platform instead of a MOOC site such as edx/udemy is because I am hoping to reach a more casual audience of people who are hoping to watch a quick, straight-to-the-point video on how to complete a particular task which is something I have often hoped to find when doing a search on Youtube for a tutorial video on how to use a particular function in a library, or how to do a particular task such as an animated/interactive chart in python. 

I set up my youtube channel called 'Practical Data Science' and I spent the day becoming more familiar with the youtube analytics page and researching if there was a way to track user behavior.

Youtube analytics provides information regarding how long on average a user watched the video to completion, average viewing time, and likes/dislikes, but so far I have been unable to find a way to track user browsing, so if a student watched the first lesson, then skipped over to the 5th lesson, this is information which I am currently unable to find a way to access.

\subsection{Further Readings}
I'm not used to the idea of picking up papers or articles with no intention of completely reading from start to finish, but after overcoming the initial overwhelming feeling, things proceeded smoothly as I moved through the first paper and then proceeded to look through the references for the next paper to read, and so forth. The rabbit hole really doesn't seem to have an end and the thought crossed my mind that it would be a fun project for the simulation class I'm also taking to write a program that picks a random article and selects a random reference within the article and continues in a loop until a duplicate article is encountered.

I was mainly focused on reading through articles that did something similar to what I wanted to try for my project, so in summary of what I found, papers like \citep{guo1} provided information for online teaching methods which had high student engagement, \citep{joyner1} ran a study on how online classes could be just as effective as on-campus by measuring student performance, \citep{wood1} provided a framework for tutoring methodology and the role which tutoring plays in society, \citep{cross1} provided a study for student preferences for handwritten versus typeface lessons and introduced a hybrid method called 'TypeRighting' while \citep{leva1} also did something similar with powerpoint slides, \citep{tackett1} analyzed the metrics from a popular educational youtube channel on medical topics, 'Osmosis' and provided insights to how open platforms like youtube can increase audience reach to low-income countries, and  \citep{lim1} talked about the saturation of videos on youtube and how more than half for the chosen topic of infection prevention were non-educational and misinformed and the importance of utilizing social media for education providers to strengthen their reach.

\citep{hew1} discussed the challenges of open courses and the high drop-out rate as well as reasons that might cause the issues. One of the topics which really stood out to me here which I also relate to is that students sometimes have a difficult time finding a resource to ask for help when they are stuck, and the effect of not receiving immediate feedback from the students make it difficult for the instructor and course creator to adequately respond or make changes. I hope to avoid both issues by utilizing the open nature of youtube to have other learners come in and provide support where my content my have left people with questions, on top of that, I am hoping that the large traffic of youtube will provide me with enough traffic to establish fast-feedback unlike the lower traffic numbers that would be coming in through a MOOC platform.

\citep{hone1} reviewed factors affecting retention rates in MOOCs and put emphasis on course content and instructor engagement being a big factor that grouped courses by drop-out rate.

\citep{kim1} provided analysis of edx courses and provided data driven conclusions on what is the optimal length, content, and structure for courses to have the highest retention rate and student engagement. 

\citep{kiz1} placed online learners into buckets, and went into great detail to provide the different paths each learner can take, what keeps them engaged, and what doesn't. These are all great data-backed conclusions which will help in designing the course.

\citep{houston1} gave a thorough report on the positive impacts that student engagement has with final outcome. The data showed that the strongest predictor of a student's grade was his/her direct engagements such as posting threads, replying to threads, and such. With this in mind, it would be an interesting idea to try to incorporate some type of medium to allow learners to engage in direct communication in a more structured setting (For example, somewhere besides the comments section of a video).

\section{Activity}
\subsection{Paper 1: How video production affects student engagement: An empirical study of MOOC videos \citep{guo1}}

\subsubsection{Need}
There is a need to understand how students engage with content in online classroom settings in order for future content creators to provide better content.
\subsubsection{Method}
The author analyzed data from 4 edX courses and supplemented their findings with interviews with 6 edX staff who were behind the creation of the courses.
\subsubsection{Audience}
The audience of this paper are online instructors and researchers, while the participants were the students who took part in the online courses which were analyzed.
\subsubsection{Results}
The authors found that shorter videos, having a talking head (instructor being visible), khan-academy style tutorial structures, and pre-planning the lesson for an online setting all provided higher engagement statistics than courses which did not have these features. 
\subsubsection{Critique}
There is very large evidence that confirms my own ancedotal experience that long lecture videos tend to be hard to follow and breaking apart lectures into sections and chunks makes things more digestible. When designing my own course, I think the key points I would like to use are having a tutorial structure, being visible to the student (But should I always wear the same outfit?), and gaining further knowledge and experience in pre-processing/lesson planning.

\subsection{Paper 2: Medical Education Videos for the World: An Analysis of Viewing Patterns for a YouTube Channel \citep{tackett1}}
\subsubsection{Need}
There is a need to understand the type of audience and reach online video platforms like Youtube have in terms of providing educational videos.
\subsubsection{Method}
The author analyzed the data from the popular medical YouTube channel 'Osmosis' between 2016-2017 to build profiles of the viewers.
\subsubsection{Audience}
The audience is anyone interested in open video platforms, online education, and/or medical topics. The participants were the viewers of the videos on the YouTube channel. 
\subsubsection{Results}
The results demonstrated that although a majority of the views were coming from world bank classified countries that were high income, there was still a significant portion that were being viewed by individuals all over the world, with a country participation rate of 213/218. Conversely, medical videos on other platforms garnered a fraction of the views compared to YT.
\subsubsection{Critique}
This study should inspire anyone thinking of creating content to consider a channel like YT to also upload their materials as the audience reach and global impact can be significantly higher. With that said, a lot of planning goes into each Osmosis video in order to avoid western jargon. Things like the metric system are used in tutorials, and collaboration with wikimedia Project Medicine provides a large amount of subtitles for each video to allow non-English speakers to understand the materials. I would like to do further research into what other channel data is available and if Osmosis data can still be retrieved. The channel at the time of the study had 100k subscribers, and today that number is over 1.3m.

\subsection{Paper 3: Pedagogy Meets PowerPoint: A Research Review of the Effects of Computer-Generated Slides in the Classroom \citep{leva1}}
\subsubsection{Need}
There is no definitive collection of research on the effectiveness of computer generated slides through presentation software such as PowerPoint.
\subsubsection{Method}
The author perused the existing research on the topic and reviewed studies along with their drawn out data and conclusions.
\subsubsection{Audience}
The audience is instructors and students.
\subsubsection{Results}
There was a large variance in data presented in the studies, but the overall trend seemed to show that computer generated presentations and materials are the future and provide better learning outcomes for the student based on surveys and assessments of students who went through more traditional lessons i.e. lecture recordings, chalk board write-ups, and hand-written notes.
\subsubsection{Critique}
This report provides strong data to indicate that offline teaching techniques do not translate in their effectiveness to online classrooms and there is a strong need to accordingly incorporate new technologies in innovative ways in order to achieve good results for when an instructor is designing their course materials.

\subsection{Paper 4: Students’ and instructors’ use of massive open online courses (MOOCs): Motivations and challenges \citep{hew1}}
\subsubsection{Need}
With the new field of MOOCs there is very little research into the driving motivations and challenges facing students on the new platform. 
\subsubsection{Method}
The authors searched through 362 abstracts with the MOOC or massive open online course keyword and selected specific papers which provided insights into the motivation of both student and teacher to participate in an online course.
\subsubsection{Audience}
The audience are other instructors and those interested in MOOCs.
\subsubsection{Results}
There were some common challenges highlighted in both the motivation and challenges portion, but I was more focused on the challenges. Student complaints included lack of incentive, engagement, feedback, and ability to seek out immediate help while instructors had the problem of lack of student response in online discussion, lack of immediate student feedback when creating video content, and inability to track student progress as effectively as on-campus.
\subsubsection{Critique}
I think the challenges highlighted here are key points to take away for course design. The next challenge for myself will be to research into methods for designing a course dynamically and finding ways to see how people respond to my content in order to see if I can identify problems in my methods early and address them and not dig myself too deep in creating an entire lesson that is poorly designed.

\subsection{Paper 5: Pass the Idea Please: The Relationship between Network Position, Direct Engagement, and Course Performance in MOOCs \citep{houston1}}
\subsubsection{Need}
There is a lot of data surrounding MOOCs but not much analysis in terms of what key features in various classes engage students the most and what features of students are most strongly correlated with classroom success.
\subsubsection{Method}
Data from 3 classes was analyzed, 2 of which were business-related and the 3rd being a Matlab course.
\subsubsection{Audience}
The intended audience would be instructors, students, and anyone interested in MOOCs.
\subsubsection{Results}
Despite the large amount of data analyzed, the authors found that the strongest predictor of a student's completion and grade was their direct participation on the communication channels such as discussion boards. 
\subsubsection{Critique}
This data reinforces the idea that active engagement of the student in the course provides the best path to success, so from the perspective of someone who is creating videos meant to be viewed by an individual, the challenge I have to address is how to create this engaging environment that has some type of tactile feedback for the student in order to keep their attention throughout the course.

Ideas I am interested to look further into involve implementing a structure where I ask interesting and fun but not necessarily challenging questions at the end of lessons with answers on the next video, providing brief assignments and surveys, and creating challenges for students to complete and upload their results as a YouTube video in my lessons comments. One of the best resources I am researching is Khan Academy, which I believe is one of the most successful platforms at providing free educational videos to anyone that is interested.

\section{References}
\printbibliography[heading=none]
\end{document}