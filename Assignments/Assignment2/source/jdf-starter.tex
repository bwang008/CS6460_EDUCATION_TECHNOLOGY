\documentclass[
	%a4paper, % Use A4 paper size
	letterpaper, % Use US letter paper size
]{jdf}

\addbibresource{ref.bib}

\author{Benjamin Wang}
\email{bwang421@gatech.edu}
\title{CS 6460 Assignment 2}
\date{January 2019}

\begin{document}
%\lsstyle

\maketitle

\section{Journal}
\subsection{Introduction}

Key words I researched this week were "coding bootcamp", "engaging youtube education", and "decentralized learning education" of which I found the papers covering coding bootcamps to be the most interesting. 

Key points of papers which I discovered highlighted the impact of technology on education, the need for universities around the world to incorporate technology in order to compete in the global market, as technology makes education cheaper and more affordable, the colleges which are only offline will no longer be able to compete in terms of cost as the gap in quality of education being offered online and on-campus begins to close.

\citep{1} discusses the rise of Big Data and the impacts it has on education as well as the assumptions that it brings. Key points from this paper discussed are that Big Data only allows research to be made on areas where the data is available, Big Data provides a more clear look into the behavior of learners and what they are \emph{actually} doing versus what they say they do such as through survey responses, and that Big Data does not work as a silver bullet to the problem of making education scalable, while we might see patterns, the motivation behind the formation of those patterns still needs to be investigated thoroughly outside of ML and data science methods such as person-to-person interactions. 

I think the author brings up some good points about the issues that can occur as we transition education more towards an online setting, I think many of the concerns are moot as they also apply to situations that do not involve machine learning and Big Data. I believe the power of Big Data combined with ML approaches have the ability to create paths for learners in the future which are custom fit to their specific needs.

Research into student data to understand factors that cause drop-out or low retention through profiling and classroom factors can be analyzed in mass to predict students who are at risk of failing, and recommend for them options which similar students in their situation took in order to find success, and also for class selection towards a degree, many times we find ourselves in situations which we are unaware of which classes to take and take the blind route of choosing classes based on their topic of study and abstract summary. With Big Data, methods like collaborative filtering can be employed to create recommendation systems to students based on their grade and rating of a class to explore other classes which similar students also did well and enjoyed. 

\citep{2} provides research data on social media engagement on the YouTube platform and what drives viewers to involve themselves with the platform. The study was done through a survey of 1143 student respondents from a University located in the Midwest and based on the responses, patterns and relationships were drawn out to make the assumptions of the study. From the study, various statistics were drawn to identify the type of people that like, dislike, comment on, and only view videos, the average length of time spent on YouTube, the age of their accounts, and motivations for visiting the website.

I think this was a very superficial study that does not provide much useful information. The first problem I found with this paper is that the sample size was extremely small for studying a platform where billions of people are drawn to visiting, the sample was also drawn from a single location so the age and bias of the sample had very low variance, and finally some of the survey questions tried to measure actions of a user such as how often they post toxic comments or flame, or if they are motivated to visit YouTube in order to seek a sense of community or self-validation. These type of characteristics and features are extremely difficult to measure through surveys as you have no method of verifying the honesty behind the responses. Overall, I think the entire study was flawed due to the method that was used to collect the data, the construction of the survey, and the audience in which the sampling occurred so no real conclusions can be drawn to it.

\citep{3} talks about cases of trying to integrate online material such as YouTube videos into the classrooms and the challenges facing them, things such as copyright infringement, avoiding inappropriate content, network security and firewall restrictions, are covered and work-arounds are discussed such as locally saving the videos to the hard-disk for replaying offline.

I think that the contents of this paper applies less to my scope of work as I am hoping to work entirely online with no physical presence, but I think the paper does a good job of highlighting the issues that arise when trying to cross the physical and virtual barrier not only in terms of technical issues, but legal ones as well. 

\citep{4} this paper talks about the implications that Web 2.0 has for education and the current impacts that the education world is facing today with the changing behavior of the technologically savvy generation. 

I found this paper very interesting because it was dated in 2009, which was the transition period between the old internet and new internet. I was amused to see the author mentioning web sites like "myspace" and "del.icio.us" which were very popular during that era, but have since faded into obscurity. Overall, I think the concerns that the author had about the need for universities to get with the times and start incorporating their offerings into the internet was on-point, and reading through the article felt like I was reviewing a time capsule of the internet from 10 years ago. 

\citep{5} interviewed 26 people who had completed or were in the process of completing bootcamps, collecting information in terms of the path that led them to bootcamps, and the quality of education which they received. The surveys seem to show that bootcamps have a negative stigma and some bootcamp graduates state that they feel like they struggled to keep up with their peers, however this could also be due to coming from a non-CS related background and degree. 

\citep{6} ran a study of bootcamps and whether they are more or a fad or disruptive technology. Surveys of former students seem to show a large majority reported an increase in their pay within 6 months of graduating, and a similarly large majority (86\%) felt satisfied with their education. This brings into question the value that high cost college degrees bring to students who are more interested in obtaining skillsets for the job market. I think this paper brings up good points that if a bootcamp is able to charge a fraction of the tuition compared to a University and finish in less time (Anywhere between 2-6 months) and provide statistics showing gainful employment for the learners, perhaps once the stigma surrounding learners as not having 'real educations' is overcome, this may become the future avenue for employees of the future to decide which career path they wish to travel down.

I think one of the big advantages of coding bootcamps and online learning is that people can dip their feet into the programs and examine the work and environment at no high cost which I feel grants more agency to students in making a more informed decision in what they want to do as adults. I recall one of the only driving factors that led me in my initial choice to study Electrical Engineering was the recommendation made by my math teacher, I had little to no idea what being an EE major involved and I'm sure I'm not the only person who was in that situation.

Reading this article led me down the path to investigate even further into boot camps. I was very interested in this topic because one of the key reasons I chose to attend the OMSCS program was due to my desire to obtain an affordable degree and lay down the foundations to transition my career from control systems engineering into the field of Data Science. One of the pathways I seriously considered before choosing Georgia Tech was a coding boot camp. 

After research, I determined that coding boot camps were aimed more for non-technical individuals working low-salary jobs hoping to transition into the tech industry, and not necessarily designed for those hoping to move into more advanced areas of SWE. 

The articles I read through provided great insight on the demographics of attendees being minorities and women, being motivated by higher pay and better work-life balance. \citep{7} \citep{9} \citep{10} \citep{11} \citep{12} \cite{13} \cite{14}

\section{Problem Statement}

\subsection{Background}
The cost of a college degree has resulted in a generation burdened with lifelong debt. Many college graduates find themselves unable to find high-paying work with non-technical degrees, deterring future students by bringing into question the value of attending college. Around the world, the demand for workers with software development skills is increasing while the supply simply cannot keep up from the graduates being produced in academia thus some companies have begun to look towards individuals who have taken alternative paths, such as attending coding boot camps or being self-taught, as potential hires. 
\subsection{General Problem Statement}
General problem is that the demand for high skilled workers with software development skills unmatched by the available supply within the workforce and there is still a hiring stigma surrounding individuals who are self-taught or boot camp graduates.
\subsection{Scholarly Support}
\citep{12} cites the Bureau of Labor Statistics projecting that there will be over 1 million unfulfilled software engineering jobs by 2020 with over 600,000 listings being posted in 2019 \citep{15}. \citep{8} and \citep{6} provides data showing that graduates of coding boot camps obtain higher salaries within 6 months of graduation and a majority are able to secure employment in the tech industry despite having a non-tech background or degree.
\subsection{Specific Problem Statement}
The high cost of higher education is preventing many capable workers from going back to school to obtain the skills required for software development and lack of adoption within the industry to hire alternative path learners has resulted in the high demand for software development worker being left unfulfilled. 
\subsection{Conclusion}
\citep{7} indicates that the majority of alternative path hires hold 4-year degrees, although not necessarily in CS. With the mindset that many companies have that 4-year degrees are mandatory to be considered for a job posting, there needs to be further studies done and data collected on learners who did not obtain a 4-year degree but are working in software development. \citep{13} presents online learning and boot camps as a potentially disruptive technology due to the number of 4-year degree holders going back to school to secure better paying work. 

\section{Research Question}

\subsection{Research Question 1}
Can free platforms like YouTube be leveraged to reach out and train viewers in the technical skills needed work in the tech industry?
\subsubsection{Sub Question 1}
To what extent can online videos cover the skills needed for coding?
\subsubsection{Sub Question 2}
What measurable or metric could be used to determine that a viewer is sufficiently trained and prepared for work? 
\subsubsection{Sub Question 3}
What is the willingness of tech companies to hire people who are entirely trained online with no traditional 4-year degree? 
\subsubsection{Justification}
\textbf{Complexity:} The question fulfills the complexity requirement because the topic of YouTube and Online Learning requires extensive research into the role that social media and video media can play in educating, and while there is already research supporting the idea that online classes can be as rigorous and provide as much knowledge when compared to on-campus courses, there is not much data on non-accredited programs. There is also a need to analyze the existing job openings in tech companies to examine the desired skills, and then examine the methodology which existing people who possess those skills acquired their knowledge, and examine the viability of transferring that knowledge through online classes to a possibly non-technical audience.

\textbf{Arguability:} All of these questions can be answered with studies and data, I believe the portion to measure company willingness to hire non-degree holders would require surveys of both recruiters and individuals who do not possess a college degree in order to get a sense of what the landscape is for this group. The other questions can leverage existing MOOC and YouTube channel data to draw conclusions regarding the viability of creating a free online program to empower individuals to obtain skills for a career transition. 

The issues then arise of how to create a program that is engaging to the audience, garners the respect of the industry, and scales globally to serve an audience outside of the west. 

\section{References}
\printbibliography[heading=none]
\end{document}